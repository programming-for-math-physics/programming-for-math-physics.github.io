演習環境のURLを受け取るためのダミーの課題です 
 * 提出不要
 * 課題のフィードバックを受け取ってください

 \documentclass[12pt,dvipdfmx]{article}
\setlength{\oddsidemargin}{-1.3truecm}
\setlength{\evensidemargin}{-1.3truecm}
\setlength{\textwidth}{18.5truecm}
\setlength{\headsep}{1truecm}
\setlength{\topmargin}{-2truecm}
\setlength{\textheight}{23truecm}
\usepackage{graphicx}
\DeclareGraphicsExtensions{.pdf}
\DeclareGraphicsExtensions{.eps}
\graphicspath{{out/}{out/tex/}{out/pdf/}{out/eps/}{out/tex/gpl/}{out/tex/svg/}{out/pdf/dot/}{out/pdf/gpl/}{out/pdf/img/}{out/pdf/odg/}{out/pdf/svg/}{out/eps/dot/}{out/eps/gpl/}{out/eps/img/}{out/eps/odg/}{out/eps/svg/}}
\usepackage{listings}
\usepackage{fancybox}
\usepackage{hyperref}
\usepackage{color}

\title{環境設定 (授業HPにたどりつくまで)}
\author{田浦}
\date{}

\pagestyle{empty}
\begin{document}
\maketitle
\thispagestyle{empty}

\begin{enumerate}
\item ITC-LMS $\leftrightarrow$ 授業ホームページ \url{https://pmp.eidos.ic.i.u-tokyo.ac.jp/}
  
\item 講義資料などは授業ホームページから
\item 外部に漏れてはいけない情報はITC-LMSとGoogle spreadsheet
\item オンライン授業は gather.town (今日試す)
  
\item PCのラベル(GCLxxx)をチェック
\item PC起動
\item ユーザ user2 (山崎クラス), user3 (田浦クラス)でログイン. 
初期パスワードはその場で教える
\item パスワードを必ず変更(他の人のPCとの取り違え防止)! 
  「ユーザアカウント」を開く (Ubuntu ボタン $\rightarrow$ user);
  ロック解除 
  $\rightarrow$ パスワードを変更
\item 一旦ログアウト (右上の電源ボタン)
\item 設定したパスワードでログインしなおし
\item Wifi wpa.c に接続されていることを確認.
  デスクトップ右上にIPアドレス (10.249.x.x みたいな数字) 
  が表示されていることを確認.
\item ブラウザ(左パネルのオレンジの firefox ボタン)を起動.
  ネットにつながっているかをチェック
\end{enumerate}

以下は毎回の手順

\begin{enumerate}
\item ゼミホームページ \url{http://pmp.eidos.ic.i.u-tokyo.ac.jp/} にアクセス.
\item 「演習環境」リンクからシートにアクセス.
\item IPアドレス記入 (毎回記入. ネットにつながったことの確認 + こちらで画面を共有するため)
\item 自分用の「演習環境URL」リンクをつついて演習環境に入る
\end{enumerate}

{\huge{\bf 次回以降わからなくならないように記入しておくこと}}

\begin{center}
  {\Large
\begin{tabular}{|l|l|}\hline
  PC番号 &  \\\hline
  PCに設定したパスワード & \\\hline
  Jupyter環境 & {\tt https://taulec.zapto.org:\underline{191\phantom{0000}}/} \\\hline
  Jupyter環境パスワード & \\\hline
\end{tabular}}
\end{center}

Jupyter環境は自宅からでもアクセスできます!


\end{document}
