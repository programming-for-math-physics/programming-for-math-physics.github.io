
%\documentclass{jsarticle}

\documentclass{article}
\usepackage{amsmath}
\usepackage[dvipdfmx]{graphicx,color}

%一段落目でインデントしたい
\usepackage{indentfirst}

\usepackage{comment}
\usepackage{ascmac}
\usepackage{otf}
\usepackage{url}
\usepackage{setspace}
\usepackage{wrapfig}

\usepackage[top=20truemm, bottom=20truemm, left=20truemm, right=20truemm]{geometry}

\usepackage{here}
%\begin{figure}[H]とかやると、強制的に位置を文章の一定位置に固定する。

\usepackage{caption}

\usepackage{listings,multicol}
\lstset{
	language={C++}, 
	basicstyle={\ttfamily\footnotesize},
		commentstyle={\small\ttfamily \color[rgb]{0,0.5,0}},
		keywordstyle={\small\bfseries \color[rgb]{0,0,1}},
		ndkeywordstyle={\small\ttfamily  \color[rgb]{0.5,0.5,0.0}}, %特殊キーワード、かな?
	alsoletter={\#},%無理やり#defineをキーワード化するため。
	morekeywords={\#define},%普通のキーワード
	morendkeywords={spawn,sync},%別の色で着色できる、特別キーワード
	showstringspaces=false
	identifierstyle={\small},
	stringstyle={\small\ttfamily}, 
	frame={tb},%singleとかで外側を線で囲む。tbで上下だけ、noneでなし
	breaklines=true,%折り返し
	columns=[l]{fixed},
	numbers=none,%これは、行番号。leftとかで左に行番号がつく
	xrightmargin=0zw,
	xleftmargin=3zw,
	numberstyle={\scriptsize},
	stepnumber=1,%行番号増分
	numbersep=1zw,
	lineskip=-0.25ex,
	morecomment=[l]{//},
	escapeinside={<@}{@>},
	tabsize=2,
}

\usepackage{subfig}

\captionsetup[subfloat]{position=bottom,
  farskip=10pt,topadjust=0pt,captionskip=10pt,
  nearskip=10pt,margin=10pt}

\renewcommand{\figurename}{Figure}
\renewcommand{\tablename}{}

%ページ全体の行間
\setstretch{1.0}
%ページ全体の段落間
\parskip=3pt plus 1pt

\title{ ペットボトルロケット 参考資料 v1.0 }
\author{ 岩崎 慎太郎 }

\begin{document}

\date{November 3, 2014}
\maketitle

%%%%%%%%%%%%%%%%%%%%%%%%%%%%%%%%%%%%%%%%%%%%%%%%%%%%%%%%%%%%%%%%%%%%%%%%%%%%%%%%%%%%%%%%%%%%%%%%%%%%%%%%%%%%
\section{ 問題 }
%%%%%%%%%%%%%%%%%%%%%%%%%%%%%%%%%%%%%%%%%%%%%%%%%%%%%%%%%%%%%%%%%%%%%%%%%%%%%%%%%%%%%%%%%%%%%%%%%%%%%%%%%%%%

ペットボトルロケットを子供と一緒に作って飛ばしに行くことを考える。
せっかく一緒に作ったので、出来るだけ遠くに飛ばして喜ばせてあげたい。
最適な水の量、圧力および打ち上げ角度はいくつか?

シミュレーションの条件は以下のように置くが、合理的な範囲で適宜条件の追加および削除をしてよい。
また、計算に必要な定数は各自で補うこと。

\begin{itemize}
\item 大気圧は標準大気圧で、気温は25度で湿度はなく、風は吹いていない
\item 空気は粘性を持たず、比熱比$1.4$の理想気体として振舞う
\item 水と空気の入るタンク(ペットボトル)は、半径$45mm$で容積$1.5L$の円柱で、耐圧0.6MPaとする
\item ノズルは直径$4mm$、ロケットの先端は頂角60度の円錐で、翼はなく、ロケットの全体の重さは150gとする
\end{itemize}


%%%%%%%%%%%%%%%%%%%%%%%%%%%%%%%%%%%%%%%%%%%%%%%%%%%%%%%%%%%%%%%%%%%%%%%%%%%%%%%%%%%%%%%%%%%%%%%%%%%%%%%%%%%%
\section{ 小問題 }
%%%%%%%%%%%%%%%%%%%%%%%%%%%%%%%%%%%%%%%%%%%%%%%%%%%%%%%%%%%%%%%%%%%%%%%%%%%%%%%%%%%%%%%%%%%%%%%%%%%%%%%%%%%%

%%%%%%%%%%%%%%%%%%%%%%%%%%%%%%%%%%%%%%%%%%%%%%%%%%%%%%%%%%%%%%%%%%%%%%%%%%%%%%%%%%%%%%%%%%%%%%%%%%%%%%%%%%%%
\subsection{ 小問題1 }
%%%%%%%%%%%%%%%%%%%%%%%%%%%%%%%%%%%%%%%%%%%%%%%%%%%%%%%%%%%%%%%%%%%%%%%%%%%%%%%%%%%%%%%%%%%%%%%%%%%%%%%%%%%%

抵抗および推力がない状態の運動が、エネルギー保存則を満たすか確認しなさい。

%%%%%%%%%%%%%%%%%%%%%%%%%%%%%%%%%%%%%%%%%%%%%%%%%%%%%%%%%%%%%%%%%%%%%%%%%%%%%%%%%%%%%%%%%%%%%%%%%%%%%%%%%%%%
\subsection{ 小問題2 }
%%%%%%%%%%%%%%%%%%%%%%%%%%%%%%%%%%%%%%%%%%%%%%%%%%%%%%%%%%%%%%%%%%%%%%%%%%%%%%%%%%%%%%%%%%%%%%%%%%%%%%%%%%%%

発射時に速度を与えたとき、抵抗および推力がなければ発射角45度で飛距離が最大になるか確認しなさい。

%%%%%%%%%%%%%%%%%%%%%%%%%%%%%%%%%%%%%%%%%%%%%%%%%%%%%%%%%%%%%%%%%%%%%%%%%%%%%%%%%%%%%%%%%%%%%%%%%%%%%%%%%%%%
\subsection{ 小問題3 }
%%%%%%%%%%%%%%%%%%%%%%%%%%%%%%%%%%%%%%%%%%%%%%%%%%%%%%%%%%%%%%%%%%%%%%%%%%%%%%%%%%%%%%%%%%%%%%%%%%%%%%%%%%%%

水および空気の噴出で得られる推力の時間変化をグラフに示しなさい。

%%%%%%%%%%%%%%%%%%%%%%%%%%%%%%%%%%%%%%%%%%%%%%%%%%%%%%%%%%%%%%%%%%%%%%%%%%%%%%%%%%%%%%%%%%%%%%%%%%%%%%%%%%%%
\subsection{ 小問題4 }
%%%%%%%%%%%%%%%%%%%%%%%%%%%%%%%%%%%%%%%%%%%%%%%%%%%%%%%%%%%%%%%%%%%%%%%%%%%%%%%%%%%%%%%%%%%%%%%%%%%%%%%%%%%%

一般に推奨されている値でシミュレーションをした時、合理的な飛距離になっているか確認しなさい。

%%%%%%%%%%%%%%%%%%%%%%%%%%%%%%%%%%%%%%%%%%%%%%%%%%%%%%%%%%%%%%%%%%%%%%%%%%%%%%%%%%%%%%%%%%%%%%%%%%%%%%%%%%%%
\subsection{ 小問題5 }
%%%%%%%%%%%%%%%%%%%%%%%%%%%%%%%%%%%%%%%%%%%%%%%%%%%%%%%%%%%%%%%%%%%%%%%%%%%%%%%%%%%%%%%%%%%%%%%%%%%%%%%%%%%%

水の量、圧力および打ち上げ角度をいくつにすると、ペットボトルロケットが最も遠くに飛ぶだろうか。

%%%%%%%%%%%%%%%%%%%%%%%%%%%%%%%%%%%%%%%%%%%%%%%%%%%%%%%%%%%%%%%%%%%%%%%%%%%%%%%%%%%%%%%%%%%%%%%%%%%%%%%%%%%%
\section{ 応用問題例 }
%%%%%%%%%%%%%%%%%%%%%%%%%%%%%%%%%%%%%%%%%%%%%%%%%%%%%%%%%%%%%%%%%%%%%%%%%%%%%%%%%%%%%%%%%%%%%%%%%%%%%%%%%%%%

%%%%%%%%%%%%%%%%%%%%%%%%%%%%%%%%%%%%%%%%%%%%%%%%%%%%%%%%%%%%%%%%%%%%%%%%%%%%%%%%%%%%%%%%%%%%%%%%%%%%%%%%%%%%
\subsection{ 応用問題例1 }
%%%%%%%%%%%%%%%%%%%%%%%%%%%%%%%%%%%%%%%%%%%%%%%%%%%%%%%%%%%%%%%%%%%%%%%%%%%%%%%%%%%%%%%%%%%%%%%%%%%%%%%%%%%%

ノズルの大きさはいくつが適切だろうか。

%%%%%%%%%%%%%%%%%%%%%%%%%%%%%%%%%%%%%%%%%%%%%%%%%%%%%%%%%%%%%%%%%%%%%%%%%%%%%%%%%%%%%%%%%%%%%%%%%%%%%%%%%%%%
\subsection{ 応用問題例2 }
%%%%%%%%%%%%%%%%%%%%%%%%%%%%%%%%%%%%%%%%%%%%%%%%%%%%%%%%%%%%%%%%%%%%%%%%%%%%%%%%%%%%%%%%%%%%%%%%%%%%%%%%%%%%

2段ペットボトルロケットについて調べ、飛距離を最大化しなさい。必要な条件は適宜設定すること。

%%%%%%%%%%%%%%%%%%%%%%%%%%%%%%%%%%%%%%%%%%%%%%%%%%%%%%%%%%%%%%%%%%%%%%%%%%%%%%%%%%%%%%%%%%%%%%%%%%%%%%%%%%%%
\section{ 運動方程式を立てるにあたって }
%%%%%%%%%%%%%%%%%%%%%%%%%%%%%%%%%%%%%%%%%%%%%%%%%%%%%%%%%%%%%%%%%%%%%%%%%%%%%%%%%%%%%%%%%%%%%%%%%%%%%%%%%%%%

%%%%%%%%%%%%%%%%%%%%%%%%%%%%%%%%%%%%%%%%%%%%%%%%%%%%%%%%%%%%%%%%%%%%%%%%%%%%%%%%%%%%%%%%%%%%%%%%%%%%%%%%%%%%
\subsection{ 基本の運動方程式 }
%%%%%%%%%%%%%%%%%%%%%%%%%%%%%%%%%%%%%%%%%%%%%%%%%%%%%%%%%%%%%%%%%%%%%%%%%%%%%%%%%%%%%%%%%%%%%%%%%%%%%%%%%%%%
運動方程式が基本の式となる。
\begin{equation}
m \frac{d \vec{v}}{d t} = \vec{F} - \vec{R_{air}} - m \vec{g}
\end{equation}

水の噴出量(単位時間当たりの質量)を$\beta$とすると、
\begin{equation}
\frac{d m}{d t} = -\beta
\end{equation}

%%%%%%%%%%%%%%%%%%%%%%%%%%%%%%%%%%%%%%%%%%%%%%%%%%%%%%%%%%%%%%%%%%%%%%%%%%%%%%%%%%%%%%%%%%%%%%%%%%%%%%%%%%%%
\subsection{ 水の推力 }
%%%%%%%%%%%%%%%%%%%%%%%%%%%%%%%%%%%%%%%%%%%%%%%%%%%%%%%%%%%%%%%%%%%%%%%%%%%%%%%%%%%%%%%%%%%%%%%%%%%%%%%%%%%%

断面積$A$の口から速度$v_{water}$で水を噴出する時、
\begin{equation}
\beta_{water} = \rho_{water} A v_{water}
\end{equation}
となる。水の噴出による反作用で得られる力は、以下のようになる。
\begin{equation}
F = \beta_{water} * v_{water}
\end{equation}

$A$は小さく、定常流れが生じるとして非粘性・圧縮性流体のベルヌーイの定理を用いると、流線上で
\begin{equation}
\frac{1}{2} v_{water}^2 + \frac{p}{\rho_{water}} - g h = const.
\end{equation}
となる。ただし$g h$はその位置で受ける重力加速度の大きさを示す。

ペットボトル内の水面を考える。表面での流速を$v_{water}'$とし、ペットボトル内の圧力を$P$とすると、ベルヌーイの定理から以下のように書ける。
\begin{equation}
\frac{1}{2} v_{water}'^2 + \frac{P}{\rho_{water}} = const.
\end{equation}

ペットボトルの噴出口を考える。ロケットの傾きを$\theta$とし、大気圧を$P_0$とすると、ベルヌーイの定理から同様に
\begin{equation}
\frac{1}{2} v_{water}^2 + \frac{P_0}{\rho_{water}} - g h cos \theta = const.
\end{equation}
となる。

これが流線上で等しいので、
\begin{equation}
\frac{1}{2} v_{water}'^2 + \frac{P}{\rho_{water}} = \frac{1}{2} v_{water}^2 + \frac{P_0}{\rho_{water}} - g h cos \theta
\end{equation}
が成り立つ。この式にペットボトル自体の断面積を$A_0$として、水面と噴出口での連続の式、
\begin{equation}
A_0 v_{water}' = A v_{water}
\end{equation}
を連立することで、水が中にある時の$F$を求めることができる(ここで、Pの変化は断熱変化として計算してよい)。

\vspace{3mm}
\hrule
\vspace{3mm}

答えは、
\begin{equation}
v_{water}' = \frac{A}{A_0} v_{water}
\end{equation}
を代入して、
\begin{equation}
v_{water}^2 =\frac{ g h cos \theta + \frac{P -P_0}{\rho_{water}} } {\frac{1}{2}\left( 1 - \frac{A^2}{A_0^2} \right) }
\end{equation}
である。

ただ$g h cos \theta$の項は、本当は$(g cos \theta + a)h $ではないかと思われる。
ロケットの加速度分、圧力(というか見かけ重力分)が増すはずだと考えられる。要検討。

\vspace{3mm}
\hrule
\vspace{3mm}

%%%%%%%%%%%%%%%%%%%%%%%%%%%%%%%%%%%%%%%%%%%%%%%%%%%%%%%%%%%%%%%%%%%%%%%%%%%%%%%%%%%%%%%%%%%%%%%%%%%%%%%%%%%%
\subsection{ 空気の推力 }
%%%%%%%%%%%%%%%%%%%%%%%%%%%%%%%%%%%%%%%%%%%%%%%%%%%%%%%%%%%%%%%%%%%%%%%%%%%%%%%%%%%%%%%%%%%%%%%%%%%%%%%%%%%%

水を全て出し切ると、速やかに空気の噴出に変わる。

断面積$A$の口から速度$v_{air}$で空気を噴出する時、以下の式が成り立つ。
\begin{equation}
\beta_{air} = \rho_{air} A v_{air}
\end{equation}
空気の噴出の反作用で得られる力は、
\begin{equation}
F_1 = \beta_{air} * v_{air}
\end{equation}
であるが、さらに、ペットボトル内外の圧力差(噴出部$P_0'$、大気圧$P_0$と置いている)から、
\begin{equation}
F_2 = A \left( P_0' - P_0 \right)
\end{equation}
が圧力推力として得られる。この2つの和、
\begin{equation}
F = F_1 + F_2
\end{equation}
が得られる推力になる。

さて、空気の粘性を無視し、$A$が十分小さく、系は外部と熱のやり取りがないとして、
定常流れを仮定すると、容器内部(圧力$P$)と噴出部($P_0'$)でベルヌーイの定理を立てると以下のようになる(空気の場合、ポテンシャルの項は不要)。
\begin{equation}
\frac{1}{2} v_{air}^2 + \left( \frac{\gamma}{\gamma - 1} \right)\frac{P_0'}{\rho_{air0}} =  \left( \frac{\gamma}{\gamma - 1} \right)\frac{P}{\rho_{air}}
\end{equation}
ただし、$\gamma$は空気の比熱比${C_p}/{C_v}$である。また、$\rho_{air0}$は噴出部密度、$\rho_{air}$は容器内密度とした。
%なんで、ポテンシャルの項が不要なんだろう...

さらに断熱過程であることから、空気が理想気体であるとして以下の式を得る。
\begin{equation}
\frac{P_0'}{\rho_{air0}^\gamma} = \frac{P}{\rho_{air}^\gamma}
\end{equation}

これを解いて、噴出速度$V$が得られる。

ただし、流速はその流体の音速$v_{airmax}$を超えず、この時噴出部の圧力は大気圧$P_0$にならない。音速は、
\begin{equation}
v_{airmax} = \sqrt{\frac{\gamma R T_0}{M}}
\end{equation}
である。ただし、$T_0$は噴出部の温度である。噴出部についての理想気体の状態方程式を代入することで、
噴出部の圧力が一定以上にならないことが分かる。噴出部の圧力は、$v_{air} < v_{airmax}$の範囲で$P_0'=P_0$に従うが、
そうでない時は、$P_0'$は$v_{air} = v_{airmax}$を満たす値になる。

$P_0'$が小さくなり、$P_0$と変わらなくなった時点で、推進力がなくなり、ペットボトルロケットは慣性のみで飛ぶようになる。

\vspace{3mm}
\hrule
\vspace{3mm}

まず$v_{airmax}$は、理想気体の状態方程式$P V = n R T$について、$\frac{n M}{V} = \rho$であることを利用して、
\begin{equation}
v_{airmax} = \sqrt{\frac{\gamma P_0'}{\rho_{air0}}}
\end{equation}
となる(噴出部の値を使うことに注意)。ここから、$v_{airmax} = v_{air}$となる条件を求める。これを解くと
\begin{equation}
P_0'=\left(\frac{2}{\gamma+1}\right)^{\frac{\gamma}{\gamma-1}}P
\end{equation}
となる。つまり、上式で与えらえる解$P_0'$が$P_0$より大きいような$P$の場合は、$v_{airmax} < v_{air}$になってしまうため、
$P_0'(\neq P_0)$は上の値を使う(当然この時$v_{air} = v_{airmax}$)。

$v_{airmax} = v_{air}$の場合、
\begin{equation}
\rho_{air0} = \rho_{air} \left(\frac{P_0'}{P}\right)^{\frac{1}{\gamma}}
\end{equation}
を使って整理すると、
\begin{equation}
v_{airmax}^2 =  \frac{2 \gamma} {\gamma + 1} \frac{P}{\rho_{air}}
\end{equation}
になるから、
\begin{equation}
F=F_1+F_2=\rho_{air} A v_{air}^2 + A (P_0' - P_0) = A \frac{2 \gamma} {\gamma + 1} P + A \left(\left(\frac{2}{\gamma+1}\right)^{\frac{\gamma}{\gamma-1}}P - P_0  \right)
\end{equation}
として解ける。

$v_{airmax} > v_{air}$の場合、$P_0'=P_0$である。
\begin{equation}
\rho_{air0} = \rho_{air} \left(\frac{P_0}{P}\right)^{\frac{1}{\gamma}}
\end{equation}
であるから、これを代入して整理すると、
\begin{equation}
v_{air}^2 = \frac{2 \gamma}{\gamma -1}\left(\frac{P}{\rho_{air}} - \frac{P_0}{\rho_{air} \left( \frac{P_0}{P} \right)^{\frac{1}{\gamma}}} \right)
\end{equation}
となり、
\begin{equation}
F=F_1+F_2=F_1+0=\rho_{air} A v_{air}^2 = A \frac{2 \gamma}{\gamma -1}\left(P - \frac{P_0}{\left( \frac{P_0}{P} \right)^{\frac{1}{\gamma}}} \right)
\end{equation}
として解ける。

\vspace{3mm}
\hrule
\vspace{3mm}

%%%%%%%%%%%%%%%%%%%%%%%%%%%%%%%%%%%%%%%%%%%%%%%%%%%%%%%%%%%%%%%%%%%%%%%%%%%%%%%%%%%%%%%%%%%%%%%%%%%%%%%%%%%%
\subsection{ ペットボトルの圧力と密度 }
%%%%%%%%%%%%%%%%%%%%%%%%%%%%%%%%%%%%%%%%%%%%%%%%%%%%%%%%%%%%%%%%%%%%%%%%%%%%%%%%%%%%%%%%%%%%%%%%%%%%%%%%%%%%

断熱過程であるから、時間変化によらずペットボトルの内部で常に
\begin{equation}
\frac{P}{\rho_{air}^\gamma} = const.
\end{equation}
が成り立つ。

水が噴出している最中は、空気が外に漏れることはなく、ボトル中の水の減少分だけ空気の体積が増えることで、圧力が減少する。

空気が噴出している間は、質量保存の式から、
\begin{equation}
d ({\rho_{air}V)} = - \rho_{air0} A v_{air} dt
\end{equation}
が成り立ち、密度$\rho_{air}$が減少するため、圧力も減少する(ここでの$V$はペットボトルの容積である)。

%%%%%%%%%%%%%%%%%%%%%%%%%%%%%%%%%%%%%%%%%%%%%%%%%%%%%%%%%%%%%%%%%%%%%%%%%%%%%%%%%%%%%%%%%%%%%%%%%%%%%%%%%%%%
\subsection{ 抗力 }
%%%%%%%%%%%%%%%%%%%%%%%%%%%%%%%%%%%%%%%%%%%%%%%%%%%%%%%%%%%%%%%%%%%%%%%%%%%%%%%%%%%%%%%%%%%%%%%%%%%%%%%%%%%%

運動中全体を通して、抗力は進行方向の逆向きに働く。ここでは空気抵抗を考える。空気の密度を$\rho_{air}$とし、
ペットボトルロケットの速度を$v$、ペットボトルの断面積を$A_0$、抗力係数を$C_D$とすると、
\begin{equation}
R = \frac{1}{2}\rho_{air}v^2 S C_D
\end{equation}
となる。抗力係数はペットボトルロケットの先端として適当なものを選べばよい。また、このロケットには羽などの余計な付属部品はないものとする。

%%%%%%%%%%%%%%%%%%%%%%%%%%%%%%%%%%%%%%%%%%%%%%%%%%%%%%%%%%%%%%%%%%%%%%%%%%%%%%%%%%%%%%%%%%%%%%%%%%%%%%%%%%%%
\subsection{ もっと詳しいモデル }
%%%%%%%%%%%%%%%%%%%%%%%%%%%%%%%%%%%%%%%%%%%%%%%%%%%%%%%%%%%%%%%%%%%%%%%%%%%%%%%%%%%%%%%%%%%%%%%%%%%%%%%%%%%%

実際は水のみの噴出から、気液二相流(チャーン流と考えられる)を経て、空気のみの噴出になるという(立式可能なのか不明)。
また断熱変化を仮定としていたが、実際は放熱も含むため、ポリトロープ過程を考えることでより正確に導くことができると考えられる(定数をどう計算するのかは知らない)。

%%%%%%%%%%%%%%%%%%%%%%%%%%%%%%%%%%%%%%%%%%%%%%%%%%%%%%%%%%%%%%%%%%%%%%%%%%%%%%%%%%%%%%%%%%%%%%%%%%%%%%%%%%%%
\section{ 参考となるWebページ }
%%%%%%%%%%%%%%%%%%%%%%%%%%%%%%%%%%%%%%%%%%%%%%%%%%%%%%%%%%%%%%%%%%%%%%%%%%%%%%%%%%%%%%%%%%%%%%%%%%%%%%%%%%%%

\begin{itemize}
\item[1.] 金子望「ペットボトルロケットの教材化」\\ \url{http://www.asahi-net.or.jp/~hy9n-knk/petindex.htm}
       \\ 大変参考になると思います。答えが載っているような気がしますが、そのまま写すと計算の雰囲気を見失う可能性があるため、
	      一度自分で導出することをお勧めします。
\item[2.] 板倉嘉哉, 本田祐基「身近な遊具等の工学的解析(その1:ペットボトルロケットの推力特性)」 \\ \url{http://mitizane.ll.chiba-u.jp/metadb/up/AA11868267/13482084_58_365.pdf}
       \\ 実験値と理論値との比較に際して、とても参考になると思います。
	      ペットボトルロケットの特性の測定を行った論文で、理論値との比較にとても役立つでしょう。
\end{itemize}


\end{document}
